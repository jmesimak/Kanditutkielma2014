% --- Template for thesis / report with tktltiki2 class ---
% 
% last updated 2013/02/15 for tkltiki2 v1.02

\documentclass[finnish]{tktltiki2}

% tktltiki2 automatically loads babel, so you can simply
% give the language parameter (e.g. finnish, swedish, english, british) as
% a parameter for the class: \documentclass[finnish]{tktltiki2}.
% The information on title and abstract is generated automatically depending on
% the language, see below if you need to change any of these manually.
% 
% Class options:
% - grading                 -- Print labels for grading information on the front page.
% - disablelastpagecounter  -- Disables the automatic generation of page number information
%                              in the abstract. See also \numberofpagesinformation{} command below.
%
% The class also respects the following options of article class:
%   10pt, 11pt, 12pt, final, draft, oneside, twoside,
%   openright, openany, onecolumn, twocolumn, leqno, fleqn
%
% The default font size is 11pt. The paper size used is A4, other sizes are not supported.
%
% rubber: module pdftex

% --- General packages ---

\usepackage[utf8]{inputenc}
\usepackage[T1]{fontenc}
\usepackage{lmodern}
\usepackage{microtype}
\usepackage{amsfonts,amsmath,amssymb,amsthm,booktabs,color,enumitem,graphicx}
\usepackage[pdftex,hidelinks]{hyperref}

% Automatically set the PDF metadata fields
\makeatletter
\AtBeginDocument{\hypersetup{pdftitle = {\@title}, pdfauthor = {\@author}}}
\makeatother

% --- Language-related settings ---
%
% these should be modified according to your language

% babelbib for non-english bibliography using bibtex
\usepackage[fixlanguage]{babelbib}
\selectbiblanguage{finnish}

% add bibliography to the table of contents
\usepackage[nottoc]{tocbibind}
% tocbibind renames the bibliography, use the following to change it back
\settocbibname{Lähteet}

% --- Theorem environment definitions ---

\newtheorem{lau}{Lause}
\newtheorem{lem}[lau]{Lemma}
\newtheorem{kor}[lau]{Korollaari}

\theoremstyle{definition}
\newtheorem{maar}[lau]{Määritelmä}
\newtheorem{ong}{Ongelma}
\newtheorem{alg}[lau]{Algoritmi}
\newtheorem{esim}[lau]{Esimerkki}

\theoremstyle{remark}
\newtheorem*{huom}{Huomautus}


% --- tktltiki2 options ---
%
% The following commands define the information used to generate title and
% abstract pages. The following entries should be always specified:

\title{Mars-mönkjiöiden navigaatiojärjestelmä punaisella planeetalla}
\author{Jerry Mesimäki}
\date{\today}
\level{Kandidaatintutkielmaan liittyvä essee}
\abstract{}

\linespread{1.5}

% The following can be used to specify keywords and classification of the paper:

\keywords{Mars-mönkijä, Field D*, GESTALT}

% classification according to ACM Computing Classification System (http://www.acm.org/about/class/)
% This is probably mostly relevant for computer scientists
% uncomment the following; contents of \classification will be printed under the abstract with a title
% "ACM Computing Classification System (CCS):"
% \classification{}

% If the automatic page number counting is not working as desired in your case,
% uncomment the following to manually set the number of pages displayed in the abstract page:
%
% \numberofpagesinformation{16 sivua + 10 sivua liitteissä}
%
% If you are not a computer scientist, you will want to uncomment the following by hand and specify
% your department, faculty and subject by hand:
%
% \faculty{Matemaattis-luonnontieteellinen}
% \department{Tietojenkäsittelytieteen laitos}
% \subject{Tietojenkäsittelytiede}
%
% If you are not from the University of Helsinki, then you will most likely want to set these also:
%
% \university{Helsingin Yliopisto}
% \universitylong{HELSINGIN YLIOPISTO --- HELSINGFORS UNIVERSITET --- UNIVERSITY OF HELSINKI} % displayed on the top of the abstract page
% \city{Helsinki}
%


\begin{document}

% --- Front matter ---

\frontmatter      % roman page numbering for front matter

\maketitle        % title page
\makeabstract     % abstract page

\tableofcontents  % table of contents

% --- Main matter ---

\mainmatter       % clear page, start arabic page numbering

\section{MER (Mars Exploration Rover)}
% Write some science here.
Referaatti artikkelista Global Planning on the Mars Exploration Rovers: Software Integration and Surface Testing, joka on katsaus siihen kuinka Mars-mönkijät Spirit ja Opportunity suunnistavat automaattisesti punaisen planeetan pinnalla. Tekstissä perehdytään lyhyesti AutoNav-järjestelmän toimintaan ja erityisesti siitä löytyvien algoritmien GESTALTin ja Field D*:in yhteistyöhön sekä siihen miksi Field D* päätettiin mönkijöiden jo laskeuduttua asentaa näihin jälkipäivityksenä kolme vuotta laukaisun jälkeen.

Marsiin lähetettiin vuonna 2003 Nasan toimesta kaksi MER-laitetta (Mars Explorarion Rover, myöhemmin mönkijä) – Spirit sekä Rover – etsimään merkkejä veden esiintymisestä punaisen planeetan pinnalta. Yksi ongelmista on mönkijöiden liikuttelu maasta käsin 26 minuuttia kestävän signaalin siirtymisestä aiheutuvan viiveen takia. Tyypillisesti laitteille lähetetään komentoja vain aamuisin kerran Marsin vuorokaudessa (myöhemmin kierto), jotka ne toteuttavat kierron aikana. Ennen iltaa niiden keräämä data taas lähetetään Maahan analysoitavaksi. Tästä johtuen mönkijöiden riittävä autonomia on kriittistä, jotta ne kykenisivät liikkumaan mahdollisimman nopeasti tieteellisesti kiinnostavien kohteiden välillä.

Mönkijöitä voidaan liikuttaa kahdella eri tavalla: sokkoajona, jossa Maahan saapuneista kuvista päätellään mihin suuntaan on turvallista ajaa ja ajaminen suoritetaan vaaroista välittämättä haluttuun kohteeseen. Toinen vaihtoehto on luovuttaa vastuu ajamisesta mönkijöissä sijaitsevalle AutoNav-järjestelmälle, joka pyrkii liikkumaan haluttuun kohteeseen ottaen kuitenkin huomioon ympäristössä sijaitsevat vaarat ja väistämään niitä. Useimmiten näitä käytetään yhdessä siten, että sokkoajoa sovelletaan niin pitkälle kuin saaduista kuvista päätellen on turvallista ja tämän jälkeen AutoNav jatkaa ajamista.

Toisinaan automaatio ei toimi ja tämä estää mönkijöitä saavuttamasta kohteitaan. Heinäkuussa 2006 molemmat saivat järjestelmäpäivityksen mukana korjauksia em. ongelmaan, joita tämä essee pääasiassa käsittelee.

\section{AutoNav-järjestelmä}

AutoNav perustuu GESTALT-algoritmiin (grid-based estimation of surface traversability applied to local terrain), jossa mönkijän stereokameroiden ottamista kuvista luodaan malli tämän välittömästä ympäristöstä. Yksi osa mallista on ruudukkopohjainen hyvyyskartta. Kartalla jokainen solu saa sen ajettavuutta kuvaava hyvyysarvon. Hankala maasto tuottaa matalan arvon ja helppo maasto korkean. Maastoa, jota mönkijä ei voi läpäistä levitetään kartalla mönkijän koon verran, jolloin laite voidaan sijoittaa pisteeksi kartalle laskentaa varten.
	
Kartoituksen jälkeen järjestelmä luo joukon kaaria, joita pitkin kulkemalla voidaan saavuttaa haluttu kohde. Mikäli kaari ei ole suora niin sille lasketaan myös kääntymispisteet. Jokaista kaarta arvioidaan kolmella kriteerillä: vaarojen välttely, kääntymisajan minimointi ja kohteen saavuttaminen. Arvion perusteella sovellus järjestää kaarien välille äänestyksen siten, että hyvyyskartalla turvallisille kaarille annetaan enemmän ääniä kuin turvattomille, kääntymispisteiden lukumäärä korreloi negatiivisesti annettujen äänien kanssa ja mönkijää lähemmäksi kohti kohdetta vievät kaaret saavat enemmän ääniä. Äänet lasketaan tämän jälkeen yhteen ja mönkijä ajaa pienen ennalta määrätyn matkan voittanutta kaarta. Edellä selitettyä prosessia toistetaan kunnes kohde saavutetaan, ennalta määrätty tavoiteaika ylittyy tai tapahtuu virhe.
	
Ongelmaksi MER-laitteiden järjestelmän ensimmäisessä versiossa muodostui tilanne, jossa mönkijä kohtasi riittävän ison kiviröykkiön. Sen kiertämiseksi olisi jouduttu valitsemaan sellainen kaari, joka ei vie mönkijää riittävän suoraan kohti kohdetta. Suoremmat kaaret taas kulkivat kivien yli. Tilanne aiheutti äänestysvaiheessa konfliktin, jossa vaaran välttäminen oli liian suuressa konfliktissa suorimman reitin haun kanssa. Järjestelmä ei antanut mönkijän liikkua mihinkään suuntaan.

\section{Field D* ja sen toiminta}

Järjestelmää vaivanneen ongelman ratkaisemiseksi siihen lisättiin Field D*-algoritmi, jolla voidaan luoda suurempia polkuja tuntemattomassa maastossa kuin mitä GESTALT pystyi tuottamaan. Field D* käyttää myös ruudukkopohjaista mallia ympäristöstään. Tässä ruudukossa jokainen solu saa arvokseen hinnan tämän ylittämiselle ja tarkoitus on etsiä ruudukosta reitti, joka minimoi solujen läpi kulkemisesta aiheutuneet kustannukset. Toisin kuin perinteisissä ruudukkopohjaisissa reitinsuunnittelualgoritmeissa, Field D* käyttää lineaarista interpolaatiota, joka mahdollistaa esimerkiksi sen, että solusta ei tarvitse poistua kulmien tai sivun keskiosan kautta vaan mikä tahansa piste solun reunalla kelpaa. Tämä vähentää suunnitellun polun pituutta ja turhien käännösten määrää verrattuna perinteisten ruudukkopohjaisten algoritmien tuottamiin polkuihin.

Field D* perustuu siis ruudukkopohjaiseen kustannuskarttamalliin, joka muistuttaa monilta osin hyvyyskarttaa. Suurin ero on kuitenkin siinä, että hyvyyskartta pitää mönkijän jatkuvasti keskipisteenään, mutta GESTALTia laajempia polkusuunnitelmia luovan Field D*:in kustannuskartta on kiinteä ja mönkijä liikkuu tämän sisällä. Myös ruudukkojen solujen arvot toimivat päinvastaisesti. Hyvyyskartalla korkea arvo merkitsee helppoa maastoa, mutta kustannuskartalla matala arvo kertoo näppärämmin ylitettävästä solusta.

Kustannuskarttaa voidaan eroavaisuuksista huolimatta kuitenkin päivittää hyvyyskartan avulla. Ensimmäisenä vaiheena kustannuskartan jokaiselle solulle annetaan alkuarvo, joka on keskivaiheilla helpon ja hankalan kuljettavuuden välillä. Mönkijän havainnoidessa ympäristöään kustannuskartan solujen arvoa voidaan päivittää sitä mukaa kun hyvyyskartan solut saavat arvoja. Todella alhainen hyvyysarvo muunnetaan kustannuskartalla erikoisarvoksi nimeltä este ja muut hyvyyskartan solujen arvot käännetään kustannuskartan arvoiksi ja skaalataan vastaamaan solun kulkemisesta aiheutunutta kustannusta. Olennaista päivittämisessä on, että solut ovat molemmilla kartoilla saman kokoisia ja niiden reunojen tulee osua maastossa samoihin kohtiin. Nyt Field D* pystyy käsittelemään kustannuskarttansa avulla aikaisempaa suurempia alueita säilyttäen samalla hyvyyskartan ominaisuudet.

Vaikka Field D* palauttaa tuloksenaan optimaalisen kuljetun reitin kustannuksen kahden pisteen välillä niin sen integroiminen AutoNav-järjestelmään tapahtuu silti helpoiten käyttämällä algoritmia äänestyspohjaisen reitinvalinnan parantamiseen. Ensimmäisenä algoritmi laskee kustannuksen jokaisen kaaren pään ja halutun kohteen välillä. Näitä arvoja käyttämällä lasketaan kaksi uutta arvoa jokaiselle kaarelle: lineaarisesti kustannuksista ääniksi skaalautuva arvo sekä monkijän etäisyyteen kohdepisteestä perustuva ääniksi muunnettava arvo. Arvot lasketaan yhteen ja tulos korvaa GESTALTin käyttämän kaaren äänestysmenetelmän kolmannen kriteerin arvon. Viimeiseksi äänet lasketaan jälleen yhteen vaarojen välttämisestä saatujen äänten sekä käännöksien määrän minimoinnista saatujen äänten kanssa, jolloin lopputuloksena on aikaisempaa huomattavasti vankempi AutoNav-järjestelmä.

Polun suunnittelu Marsin pinnalla pienessä mittakaavassa onnistuu hyvin GESTALTilla, mutta haastavampien esteiden ohittamiseksi tämän avuksi on valjastettava laajempia alueita hallitseva Field D*. Kyseistä algoritmia on testattu kattavasti Maan pinnalla ja sen toimivuus on validoitu Marsissa. Field D* tarjoaa pelkkää GESTALTia vastaavan ja usein jopa paremman vaarojen ohi navigoinnin, joka myös estää mönkijää jäämästä jumiin kompleksien esteiden tullessa vastaan.\cite{marsjuttuja}


% --- References ---
%
% bibtex is used to generate the bibliography. The babplain style
% will generate numeric references (e.g. [1]) appropriate for theoretical
% computer science. If you need alphanumeric references (e.g [Tur90]), use
%
% \bibliographystyle{babalpha-lf}
%
% instead.

\bibliographystyle{babplain-lf}
\bibliography{references-fi}


% --- Appendices ---

% uncomment the following

% \newpage
% \appendix
% 
% \section{Esimerkkiliite}

\end{document}
