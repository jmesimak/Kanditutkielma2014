% --- Template for thesis / report with tktltiki2 class ---
% 
% last updated 2013/02/15 for tkltiki2 v1.02

\documentclass[finnish]{tktltiki2}
\nonstopmode
% tktltiki2 automatically loads babel, so you can simply
% give the language parameter (e.g. finnish, swedish, english, british) as
% a parameter for the class: \documentclass[finnish]{tktltiki2}.
% The information on title and abstract is generated automatically depending on
% the language, see below if you need to change any of these manually.
% 
% Class options:
% - grading                 -- Print labels for grading information on the front page.
% - disablelastpagecounter  -- Disables the automatic generation of page number information
%                              in the abstract. See also \numberofpagesinformation{} command below.
%
% The class also respects the following options of article class:
%   10pt, 11pt, 12pt, final, draft, oneside, twoside,
%   openright, openany, onecolumn, twocolumn, leqno, fleqn
%
% The default font size is 11pt. The paper size used is A4, other sizes are not supported.
%
% rubber: module pdftex

% --- General packages ---

\usepackage[utf8]{inputenc}
\usepackage[T1]{fontenc}
\usepackage{lmodern}
\usepackage{microtype}
\usepackage{amsfonts,amsmath,amssymb,amsthm,booktabs,color,enumitem,graphicx,listings}
\usepackage[pdftex,hidelinks]{hyperref}

% Automatically set the PDF metadata fields
\makeatletter
\AtBeginDocument{\hypersetup{pdftitle = {\@title}, pdfauthor = {\@author}}}
\makeatother

% --- Language-related settings ---
%
% these should be modified according to your language

% babelbib for non-english bibliography using bibtex
\usepackage[fixlanguage]{babelbib}
\selectbiblanguage{finnish}

% add bibliography to the table of contents
\usepackage[nottoc]{tocbibind}
% tocbibind renames the bibliography, use the following to change it back
\settocbibname{Lähteet}

% --- Theorem environment definitions ---

\newtheorem{lau}{Lause}
\newtheorem{lem}[lau]{Lemma}
\newtheorem{kor}[lau]{Korollaari}

\theoremstyle{definition}
\newtheorem{maar}[lau]{Määritelmä}
\newtheorem{ong}{Ongelma}
\newtheorem{alg}[lau]{Algoritmi}
\newtheorem{esim}[lau]{Esimerkki}

\theoremstyle{remark}
\newtheorem*{huom}{Huomautus}


% --- tktltiki2 options ---
%
% The following commands define the information used to generate title and
% abstract pages. The following entries should be always specified:

\title{Mars-luotainten reitinhakualgoritmit}
\author{Jerry Mesimäki}
\date{\today}
\level{Kandidaatintutkielma}
\abstract{Tiivistelmä.}

% The following can be used to specify keywords and classification of the paper:

\keywords{avainsana 1, avainsana 2, avainsana 3}

% classification according to ACM Computing Classification System (http://www.acm.org/about/class/)
% This is probably mostly relevant for computer scientists
% uncomment the following; contents of \classification will be printed under the abstract with a title
% "ACM Computing Classification System (CCS):"
% \classification{}

% If the automatic page number counting is not working as desired in your case,
% uncomment the following to manually set the number of pages displayed in the abstract page:
%
% \numberofpagesinformation{16 sivua + 10 sivua liitteissä}
%
% If you are not a computer scientist, you will want to uncomment the following by hand and specify
% your department, faculty and subject by hand:
%
% \faculty{Matemaattis-luonnontieteellinen}
% \department{Tietojenkäsittelytieteen laitos}
% \subject{Tietojenkäsittelytiede}
%
% If you are not from the University of Helsinki, then you will most likely want to set these also:
%
% \university{Helsingin Yliopisto}
% \universitylong{HELSINGIN YLIOPISTO --- HELSINGFORS UNIVERSITET --- UNIVERSITY OF HELSINKI} % displayed on the top of the abstract page
% \city{Helsinki}
%


\begin{document}

% --- Front matter ---

\frontmatter      % roman page numbering for front matter

\maketitle        % title page
\makeabstract     % abstract page

\tableofcontents  % table of contents

% --- Main matter ---

\mainmatter       % clear page, start arabic page numbering

\section{Ensimmäinen luku}
% Write some science here.
Tiedettä tekstimuodossa ja muutama kuva.\newline
Esimerkkilause ja lähdeviite~\cite{esimerkki2}.

\section{Globaalin reitinsuunnittelun toteutus Field D*-algoritmilla}

\subsection{Kustannusarvion parantaminen interpoloinnin avulla}
Algoritmin perustana toimii metodi, jossa jokaisesta ruudukossa sijaitsevasta solmusta lasketaan halvin mahdollinen kustannusarvio haluttuun kohdepisteeseen. Perinteisesti ruudukkopohjaisessa reitinsuunnittelussa on käytetty seuraavaa kaavaa:

\[g(s) = \min_{s'\in nbrs(s)} (c(s, s') +g(s')),\]
missä \(nbrs(s)\) on joukko kaikista solmun \(s\) naapureista, \(c(s,s')\) on kustannus kulkemiseen kaaren \(s\) ja \(s'\) välillä, sekä \(g(s')\) on kustannusarvio solmulle \(s'\).

Kaavassa oletetaan, että solmusta \(s\) voidaan siirtyä tämän naapureihin ainoastaan suoraa linjaa pitkin, joka taas johtaa aikaisemmin mainittuun ongelmaan muodostaa parhaita mahdollisia reittejä robotin rajoittuneen suuntaamisen takia. Tämä voitaisiin korjata sijoittamalla \(s'\):n tilalle \(s_b,\) jossa \(s_b\) on mikä tahansa piste solmuun liittyvän ruudun reunalla. Näitä pisteitä on kuitenkin ääretön määrä, joka tekee jokaisen pisteen laskennasta mahdotonta.

Muokkaamalla verkkoa voimme silti muodostaa approksimaation jokaiselle pisteelle \(s_b\) käyttäen lineaarista interpolointia. Sen sijaan, että solmu sijoitettaisiin ruudun keskelle, asetetaan solmu jokaisen ruudun kulmaan ja täten kaaret kulkevat ruudukon reunoja pitkin. Nyt yhden kaaren kustannus voidaan valita siten, että se on pienempi niistä kahdesta ruudusta joiden välillä se kulkee.

Tämä muokkaus johtaa siihen, että paras mahdollinen reitti kulkee jotkin kaksi naapurisolmua $\overrightarrow{s_1s_2}$ yhdistävän kaaren läpi. Solmulle \(s\) voidaan nyt laskea kustannusarvio, kun reitti kulkee sen ruudun läpi em. kaarelle. Laskemista varten tarvitaan arviot solmujen \(s1\) ja \(s2\) sekä keskimmäisen ruudun \(c\) ja alemman ruudun \(b\) kustannuksista.

Kustannusarvion tuottamiseen käytetään vielä oletusta, että kustannusarvio mille tahansa pisteelle \(s_y\), joka sijaitsee kaarella $\overrightarrow{s_1s_2}$, on funktioiden \(g(s1)\) ja \(g(s2)\) lineaarikombinaatio:

\[g(s_y) = yg(s_2)+(1-y)g(s_1),\]
missä \(y\) on etäisyys \(s_1\):stä \(s_y\):hyn. Tulee kuitenkin huomata, että \(s_y\) ei välttämättä ole em. funktioiden lineaarikombinaatio, mutta tämän oletuksen tuottama approksimaatio toimii käytännässö riittävän hyvin kun halutaan muodostaa suljettu muoto solmun \(s\) kustannusarvion palauttavalle funktiolle.

Approksimaation perusteella solmun \(s\) kustannus kun tiedetään \(s1\), \(s2\), ruutukustannukset \(c\) ja \(b\) voidaan laskea seuraavasti:

\[\min_{x,y}[bx+c\sqrt{(1-x)^2+y^2}+g(s_2)y+g(s_1)(1-y)],\]
missä $x \in [0,1]$ on solmusta $s$ alareunaa pitkin kuljettu matka kunnes käännytään ruudun yli kohti oikeaa reunaa pisteeseen, joka on $y \in [0,1]$ etäisyyden päässä solmusta $s_1$.

Tehdään vielä oletus, että \((x^*, y^*)\) ovat \(x\):n ja \(y\):n arvot, joilla ylläoleva kaava saadaan ratkaistua. Lineaarisen interpoloinnin johdosta toinen arvoista on joko yksi tai nolla. Mikäli kustannus liikkua ruudun \(c\) yli on pienempi kuin ruudun reunoja pitkin kulkeminen niin halvin reitti halkaisee ruudun \(c\) ja täten joko \(x^* = 0\) tai \(y^* = 1\). Jos taas polku ei halkaise ruutua \(c\) niin \(y^* = 0\). Täten polku on kulkee solmusta \(s\) suoraan alareunaa pitkin kohtia solmua \(s1\), siirtyy jonkin matkan \(x\) alareunalla ja leikkaa tämän jälkeen ruudun halki suoraan solmuun \(s2\), tai halkaisee ruudun \(c\) kulkemalla suoraan solmusta \(s\) johonkin oikean reunan pisteeseen \(s_y\). Halvin polku riippuu \(c\):n ja \(b\):n koosta, sekä \(s_1\):n ja \(s_2\):n kustannuserosta \(f = g(s_1) - g(s_2)\). Mikäli \(f < 0\) niin paras mahdollinen polku on 1. tapaus, jos taas \(f = b\) niin polun kustannus kulkien jonkin matkan alareunaa on yhtäpitävä sen kanssa, että alareunaa ei kuljeta ollenkaan. Jälkimmäisestä polusta voidaan ratkaista kustannuksen minimoiva \(y\) seuraavasti.

Olkoon $k = f = b$. Kustannus solmusta $s$ kaaren $\overrightarrow{s_1s_2}$ lävitse on

\[c\sqrt{1+y^2}+k(1-y)+g(s2).\]
Jossa kustannuksen derivaatasta suhteessa \(y\):hyn ja asettamalla se nollaksi saadaan
\[y^* = \sqrt{\frac{k^2}{(c^2-k^2)}}\]

Lopputulos on sama huolimatta siitä kuljetaanko alareunaa pitkin, joten merkitseväksi tekijäksi jää se kumpaa reunaa kulkeminen tulee halvemmaksi. Mikäli \(f < b\) niin käytetään oikeaa reunaa ja lasketaan polun kustannus arvolla \(k = f\). Jos taas \(b < f\) niin käytetään alareunaa jolloin \(k = b\) ja \(y^* = 1 - x^*\). Täten algoritmi halvimman polun laskemiseen solmusta \(s\) mihin tahansa pisteeseen kaarelle, joka sijaitsee vierekkäisten naapurien \(s_a\) ja \(s_b\) välissä laskemiseen on seuraava:

\begin{lstlisting}[mathescape=true]
ComputeCost($s,s_a,s_b$)
if ($s_a$ on solmun $s$ diagonaalinaapuri)
	$s_1 = s_b;$
	$s_2 = s_a;$
else
	$s_1 = s_a;$
	$s_2 = s_b;$

$c$ on kustannus ruudulle, jonka kulmat ovat $s, s_1, s_2$;
$b$ on kustannus ruudulle, jonka kulmat ovat $s, s_1$, mutta ei $s_2$;

if ($\min(c,b) = \infty$)
	$v_s = \infty;$
else if ($g(s_1) \leq g(s_2)$)
	$v_s = \min(c,b) + g(s_1);$
else
	$f = g(s_1) - g(s_2);$
	if ($f \leq b$)
		if ($c \leq f$)
			$v_s = c\sqrt{2} + g(s_2);$
		else
			$y = \min(\frac{f}{\sqrt{c^2-f^2}}, 1);$
			$v_s = c\sqrt{1+y^2}+f(1-y)+g(s_2);$
	else
		if($c \leq b$)
			$v_s = c\sqrt{2}+g(s_2);$
		else
			$x = 1-\min(\frac{b}{\sqrt{c^2-b^2}},1);$
			$v_s = c\sqrt{1+(1-x)^2}+bx+g(s_2);$
return $v_s$;

\end{lstlisting}
% --- References ---
%
% bibtex is used to generate the bibliography. The babplain style
% will generate numeric references (e.g. [1]) appropriate for theoretical
% computer science. If you need alphanumeric references (e.g [Tur90]), use
%
% \bibliographystyle{babalpha-lf}
%
% instead.

\bibliographystyle{babplain-lf}
\bibliography{references-fi}


% --- Appendices ---

% uncomment the following

% \newpage
% \appendix
% 
% \section{Esimerkkiliite}

\end{document}
